\documentclass[10pt,handout]{beamer}
\usepackage{xmpincl}
\includexmp{license}
\usepackage[T1]{fontenc}
\usepackage[latin1]{inputenc}
\usepackage{ae}
\usepackage[french]{babel}
\usepackage{array, longtable}
\usepackage{tikz}
\usetikzlibrary{arrows,shapes,trees,fit,decorations.pathmorphing,backgrounds}
\usetheme{Antibes}
\setbeamertemplate{navigation symbols}{}

% for printing
\usepackage{pgfpages}
\pgfpagesuselayout{2 on 1}[a4paper,border shrink=5mm]
%\pgfpagesuselayout{resize to}[a4paper,border shrink=5mm,landscape]


\usepackage{listings}
\usepackage{verbatim}
\makeatletter
\newwrite\lstvrb@out
\newenvironment{code}{%
  \begingroup
  \@bsphack
  \immediate\openout\lstvrb@out\jobname.lst
  \let\do\@makeother\dospecials\catcode`\^^M\active
  \def\verbatim@processline{%
    \immediate\write\lstvrb@out{\the\verbatim@line}}%
  \verbatim@start}{%
  \immediate\closeout\lstvrb@out
  \@esphack
  \endgroup
  
  \begin{alertblock}{}
    \lstinputlisting[language=java]{\jobname.lst}
  \end{alertblock}}
\makeatother 

\lstset{language=java, basicstyle=\small, commentstyle=\color{blue}\textrm}

\title{Computer Science Introductory Course MSc - Introduction to Java}

\subtitle{Lecture 2: Object Oriented Programming}

\author[Pablo Oliveira]{Pablo Oliveira \texttt{<pablo@sifflez.org>}}
\institute{ENST}

\date{}

\begin{document}
\begin{frame}
  \titlepage
\end{frame}

\begin{frame}
  \frametitle{Outline}
  \tableofcontents
\end{frame}

\begin{frame}[fragile]
  \frametitle{Introduction: Object Oriented Programming}
  \begin{itemize}
    \item In the last lecture we learned that we can structure programs using 
          objects of many classes.
    \item In this lecture we will examine OOP concepts in more detail:
      \begin{description}
        \item[constructors:] creating new objects. 
        \item[references:] designating objects.
        \item[inheritance:] creating families of classes.
        \item[encapsulation:] hiding implementation.
        \item[polymorphism:] factorizing common behaviours.
        \item[interfaces:] behavioral contracts.
      \end{description}
    \end{itemize}
\end{frame}


\begin{frame}[fragile]
\frametitle{Constructors: creating a new object}

\begin{definition}
  Constructors are special methods that are called to create a new instance of their class.
\end{definition}

\begin{code}
  class BankAccount {
    int balance;
    BankAccount () {
      balance = 0;  
    }
    BankAccount (int initialDeposit){
      balance = initialDeposit;
    }
  }

  account1 = new BankAcconut();
  account2 = new BankAccount(100);
\end{code}
\end{frame}

\section{References}
\begin{frame}
  \frametitle{Outline}
  \tableofcontents[currentsection]
\end{frame}

\begin{frame}[fragile,shrink]
  \frametitle{References}
  \begin{itemize}
    \item When a variable is assigned a primitive type it contains a value.
    \item When assigned an object, array or string, it contains a reference to the data.
    \item If a is copied or passed, old and new references point to the \alert{same original object}.
  \end{itemize}

  \begin{code}
static void changeValues (int anArray[], int value){
  anArray[0] = 42;
  value = 42;
}
public static void main (String args[]){
  int v = 0; int[] a = {0,0};
  System.out.println(v + " " + a[0] + " " + a[1]);
  changeValues(a,v);
  System.out.println(v + " " + a[0] + " " + a[1]);
} 
output : 
0 0 0
0 42 0
  \end{code}
\end{frame}

\begin{frame}[fragile]
  \frametitle{Immutability}
  \begin{itemize}
    \item String are a special case, because they are immutable (cannot be changed).
    \item When you change a String a new different String is created and the characters of the orignal one are copied.
    \item For performance : do not build a string with concatenation, use StringBuilder.
  \end{itemize}
  \begin{code}
public static void main (String args[]) {
  String s1 = "hello";
  String s2 = s1;
  s1 = s1 + "!";
  System.out.println(s1 + " " + s2);
}

output : 
hello! hello
  \end{code}
\end{frame}

\section{Inheritance}
\begin{frame}
  \frametitle{Outline}
  \tableofcontents[currentsection]
\end{frame}
\begin{frame}[fragile]
  \frametitle{Inheritance}
  \structure{Q: Remember our turtle ? It could turn and advance. But we want a new class Crab that advances sideways ...}
  \begin{itemize}
    \item We could write a new class Crab, but there would a lot of code in common with Turtle (which makes the code base
          difficult to maintain).
    \item We are going to use \alert{inheritance}.
    \item Inheritance makes it possible to create a \alert{subclass} that inherits the properties of its ancestor or \alert{superclass}.
  \end{itemize}
  \only<1| handout:0>{
  \begin{center}
  \begin{tikzpicture}
    [edge from parent fork down, ->, level distance=1cm, 
     class/.style={fill=green!20,rounded corners}]
    \node[class] at(0,0) {Animal}
      child {node[class]{Turtle}}
      child {node[class]{Crab}
        child[edge from parent/.style={}]{node[fill=green!0, rounded corners, color=black!0]{RedCrab}}
      }; 
    \node at (3,0) {superclass}
      child {node{subclass}}
      ;
    \end{tikzpicture}
  \end{center}
  }
  \only<2| handout:1>{
  \begin{center}
  \begin{tikzpicture}
    [edge from parent fork down, ->, level distance=1cm, 
     class/.style={fill=green!20,rounded corners}]
    \node[class] at(0,0) {Animal}
      child {node[class]{Turtle}}
      child {node[class]{Crab}
        child{node[class]{RedCrab}}
      }; 
    \node at (3,0) {superclass}
      child {node{subclass}}
      ;
    \end{tikzpicture}
  \end{center}
  }
\end{frame}

\begin{frame}[fragile, shrink]
  \frametitle{Inheritance}
  \begin{code}
class Animal {
  Color color;
  Position position;
  double rotation;
    
  void turn(double angle) {};
  void advance() {};
}

class Crab extends Animal{
  void advance() {
    /* code for moving sideways */
  }
}

Crab crab = new Crab();
crab.color =  Color.BLUE;
crab.advance();
  \end{code}
\end{frame}

\begin{frame}[fragile]
\frametitle{overriding and hiding}
\structure{What we just did with method advance is called \alert{overriding}.}
\begin{itemize}
  \item When we call \verb!crab.advance()! the crab's advance is called!
  \item The animal's advance has been overrided.
  \item If a method is not overriden, the superclass' is used (here \verb!crab.turn(10);! would call Animal's turn implementation.
  \item the \alert{\verb!final!} keyword in a method declaration indicates that the method cannot be overridden.
\end{itemize}

\structure{overriding a static method or a variable is called \alert{hiding}, because the new static implementation or variable \emph{hides} the old one, doing this is usually a bad idea.}
\end{frame}

\begin{frame}[fragile,shrink]
\frametitle{this and super}
  \begin{itemize}
    \item for a given class \alert{\verb!this!} represents the current class and \alert{\verb!super!} the superclass. 
    \item super is used to call overriden superclass' methods.
  \end{itemize}
  \begin{code}
class Animal {
  void advance();    
}

class Crab extends Animal{
  String name;
  advance() {
    this.turn(90);
    super.advance();
    this.turn(-90);
  }
}
\end{code}
\end{frame}

\begin{frame}[fragile, shrink]
  \frametitle{Inheritance and Constructors}
  \begin{itemize}
    \item In java all the classes are subclasses of the Object class. 
    \item A subclass constructor will always call a superclass constructor.
    \item If a class possess no constructor, an empty one with no parameters is implicit.
    \item Every constructor of a subclass call the no-parameters superclass constructor.
    \item But we can control this with \verb!super! and \verb!this! keywords.
  \end{itemize}
 \begin{center}
  \begin{tikzpicture}
    [edge from parent fork down, <-, 
     sibling distance=3cm,
     level distance=1cm,
     level distance=1cm,
     class/.style={fill=green!20,rounded corners}]
     \node[class] {Object()}
     child{ node[class]{Animal()}
      child{node[class]{Crab()}
        child {node[class]{RedCrab(String name)}}
      }
      child{node[class]{Crab(String name)}
      }};
    \end{tikzpicture}
  \end{center}
\end{frame}

\begin{frame}[fragile,shrink]
  \begin{code}
class Animal {
  Position position;
  double rotation;
    
  Animal(Position position, double rotation) {
    this.position = position;
    this.rotation = rotation;
  }
}

class Crab extends Animal{
  String name;
  Crab(Position position) {
    super(position, 90);
  }
  Crab(Position position, String name) {
    this(position);
    this.name = name;
  }
}
\end{code}
\end{frame}

\begin{frame}[fragile]
  \frametitle{abstract methods}

  \structure{Suppose we add birds to our class hierarchy.}

  \begin{itemize}
    \item birds and crabs do not move the same way... there is no common implementation for advance that we can put in Animals.
    \item we could create an empty \verb!advance()! in the \verb!Animal! class and override it in \verb!Bird! and \verb!Crab!.
    \item Yet, another programer could add a new subclass and forget to implement the \verb!advance()! method.
    \item Thus, we use \alert{abstract methods}.
  \end{itemize}

  \begin{definition}
    \begin{itemize}
      \item An abstract method is a method which has no implementation. 
      \item An abstract class is a class with abstract methods.
      \item It is mandatory for all the non-abstract subclasses to override all the abstract methods.
      \item An abstract class cannot be instantiated.
    \end{itemize}
 \end{definition}
\end{frame}

\begin{frame}[fragile,shrink]
  \begin{code}
abstract class Animal {
  Position position;
  double rotation;

  abstract void advance();    
 
}

class Crab extends Animal{
  String name;
  void advance() {
    /* crab moves */
  }
}

Animal a = new Animal(); // COMPILATION ERROR
Crab   c = new Crab();   // Works!
\end{code}
\end{frame}

\section{Encapsulation}
\begin{frame}
  \frametitle{Outline}
  \tableofcontents[currentsection]
\end{frame}

\begin{frame}
  \frametitle{Encapsulation}
  \begin{definition}
  Encapsulation is the act of hiding properties and methods inside a class.
  \end{definition}
  \begin{itemize}
    \item This allows to protect classes from unexpected side-effects from the outside.
    \item It also enforces implementation agnostic programming, which is a good idea.
  \end{itemize}
\end{frame}

\begin{frame}[fragile]
  \frametitle{Packages}
  \begin{definition}
  \begin{itemize}
    \item A \alert{package} is a group of classes.
    \item Packages define a \alert{namespace}.
    \item Classes in the same package share the same \alert{namespace}. 
  \end{itemize}
  \end{definition}
  \begin{code}
    package Animals;
    class Animal{}
    class Crab{}

    import Animals.Crab;
    import Animals.*;
    class MyProgram{}
  \end{code}
\end{frame}

\begin{frame}[fragile]
  \frametitle{Acces modifiers}
  \structure{In java encapsultation is obtained through acces/visibility modifiers}.
  \begin{itemize}
    \item Classes can be \alert{\verb!public!}, visible by everyone \newline
          or \alert{without modifier} in which case they are only visible inside their package (a group of classes).
    \item Class members (variables and methods) can have 4 modifiers with different degrees of visibility.
    \newline
    \begin{longtable}{|c|c|c|c|c|}
    \hline
    Modifier    & Class & Package & Subclass & World \\
    \hline
    public      & Y     & Y       & Y        & Y \\
    \hline
    protected   & Y     & Y       & Y        & N \\
    \hline
    no modifier & Y     & Y       & N        & N \\
    \hline
    private     & Y     & N       & N        & N \\
    \hline
    \end{longtable}
  \end{itemize}
\end{frame}

\begin{frame}[fragile]
\begin{code}
packages animals;
class Animal {
  private double rotation;
  public void turn(double angle)
    {position += angle;} 
}
class Crab extends Animal {
  public void turnBack() {
    turn(180);       // legal
    rotation += 180; // illegal
  }
}
\end{code}
\end{frame}

\section{Polymorphism}
\begin{frame}
  \frametitle{Outline}
  \tableofcontents[currentsection]
\end{frame}

\begin{frame}[fragile,shrink]
  \structure{Q: How to make a group of animals advance ?}
  \begin{itemize}
    \only<1>{
    \item We want to make a group of animals (crabs and turtles) advance at the same time.
    \item We need a container for all of them, what is the container type ?
    }
    \only<2>{
    \item \alert{\huge{Nightmare}}
    }
  \end{itemize}
    \begin{code}
  int numberCrabs; int numberTurtles;
  Crab[] crabs;
  Turtle[] turtles;
  
  moveAllAnimals () {
    for(int i=0; i < numberCrabs; i++)
      crabs[i].advance();
    for(int i=0; i < numberTurtles; i++)
      turtles[i].advance();
  }
  void addCrab (Crab c) {crabs[numberCrabs++]=c;}
  void addTurtle (Turtle t) 
    {turtles[numberTurtles++]=t;}
  
  addCrab(new Crab());
  addTurtle(new Turtle());
    \end{code}
\end{frame}

\begin{frame}[fragile]
  \frametitle{Polymorphism}
  \structure{Use Polymorphism, or the capacity to treat an instance as one of its super classes}

  \begin{code}
  int numberAnimals;
  Animal[] animals;
  void moveAllAnimals(){
    for (int i=0; i < numberAnimals; i++)
      animals[i].advance();
  }
  void addAnimal(Animal a)
    {animals[numberAnimals++] = a;} 
  
  addAnimal(new Crab());
  addAnimal(new Turtle());
  \end{code}

\begin{itemize}
    \item \alert{\huge{Better}}
\end{itemize}
\end{frame}

\begin{frame}[fragile]
\frametitle{Dynamic and Static type: Casts}
\begin{code}
  Animal animal;
  animal = new Crab();  
\end{code}
\begin{description}
\item[static type] Animal
\item[dynamic type] Crab
\end{description}
\begin{itemize}
  \item when calling an instance method the dynamic type is used.
  \item when calling a static method the static type is used.
  \item you can change the dynamic type (only to super-classes of the instanciation type) using casts:
  \begin{code}
    Crab c = (Crab) animal; // OK
    Turtle t = (Turtle) animal; // Runtime ERROR
  \end{code}
\end{itemize}
\end{frame}

\begin{frame}[fragile]
\frametitle{Dynamic dispatching}
\begin{itemize}
  \item When you call an instance method, the method used is the one provided by the dynamic class,
        this is called \alert{dynamic dispatching}.
  \item It is the really powerful idea behind polymorphism: 
  \begin{itemize}
    \item You can treat a group of objects the same way
    \item When you do an operation on one of the objects, the adequate operation will be chosen depending on the
          dynamic type of the object.
  \end{itemize}
\end{itemize}
\end{frame}
\section{Interfaces}
\begin{frame}
  \frametitle{Outline}
  \tableofcontents[currentsection]
\end{frame}

\begin{frame}[fragile]
  \frametitle{Multiple inheritance ?}
  \begin{itemize}
    \item We have added further classes to our animal class hierarchy:
      Swimming with method \verb!swim()!, Walking with
      method \verb!walk()!.
     \item As our turtle can both swim and walk we would like it to
       inherit from both classes.
     \item But in java this is \alert{forbidden}.
  \end{itemize}
\end{frame}

\begin{frame}[fragile]
  \frametitle{Multiple inheritance : problem}
  \begin{center}
  \begin{tikzpicture}
    [classbox/.style={rectangle, draw=black!50, rectangle split,
      rectangle split parts=2, node distance=3cm}]
    \node[classbox](A)  {\verb!class A!\nodepart{second}\verb!A.doIt();!};
    \node[below left of=A,classbox] (B1)
    {\verb!class B1!\nodepart{second}\verb!override: B1.doIt();!} edge[<-] (A);
    \node[below right of=A,classbox] (B2)
    {\verb!class B2!\nodepart{second}\verb!override: B2.doIt();!} edge[<-] (A);
    \node[below left of=B2,classbox] (C) {\verb!class C!\nodepart{second}\verb!doIt() ?!} edge[<-] (B2) edge[<-] (B1);
  \end{tikzpicture}
  \end{center}
  \alert{When we call \verb!doIt()! on C, do we call $B1$ or $B2$
    implementation?}
  \pause
  \structure{Multiple answers to this problem (see for example Eiffel's
    nice solution), Java Answer: \alert{Interfaces}}.
\end{frame}

\begin{frame}[fragile]
  \frametitle{Interfaces}
  \begin{definition}
    An interface is a \alert{behavioural contract} that a class
    decides to honor.
    \begin{itemize}
    \item Concretely, an interface is a collection of method signatures.
    \item If a class \alert{\verb!implements!} an interface, it has to
      provide a body for each of those methods.
    \item A class can implement \structure{multiple} interfaces.
    \item An interface can extend another (single) interface.
     \end{itemize}
 \end{definition}
 \pause
 \structure{Q: Why does it solves the multiple inheritance problem ?}
 \structure{A: We multiply interface, we do not multiply implementation...}
\end{frame}

\begin{frame}[fragile]
  \frametitle{Multiple inheritance with interfaces}
  \begin{center}
  \begin{tikzpicture}
    [classbox/.style={rectangle, draw=black!50, rectangle split,
      rectangle split parts=2, node distance=3cm}]
    \node[classbox](A)  {\verb!interface A!\nodepart{second}\verb!doIt();!};
    \node[below left of=A,classbox] (B1)
    {\verb!interface B1!\nodepart{second}\verb!doIt();!} 
    edge[<-] node{extends} (A);
    \node[below right of=A,classbox] (B2)
    {\verb!interface B2!\nodepart{second}\verb!doIt();!} 
    edge[<-]
    node{extends} (A);
    \node[below left of=B2,classbox] (C) 
    {\verb!class C!\nodepart{second}\verb!C.doIt()!} 
    edge[<-, dotted] node{implements} (B2)
    edge[<-, dotted] node{implements} (B1);
  \end{tikzpicture}
  \end{center}
    \structure{B2 and B1 asked for a method \verb!doIt!, C provides
      it, no ambiguity}
\end{frame}

\begin{frame}[fragile]
  \begin{code}
  public interface Swimming {
    void swim();
  }
  public interface Walking {
    void walk();
  }
  class Turtle extends Animal 
               implements Swimming,Walking{
    void swim() { /* swim implementation */}
    void walk() { /* walk implementation */}
  }
  \end{code}
\end{frame}

\section{Summary}
\begin{frame}
  \frametitle{Summary}
  
  \begin{itemize}
    \item To factorize code, creating classes hierarchies is
      important.
    \item Each class should hide its implementation to make code
      robust and maintainable.
    \item With polymorphism one can design elegant, factorised code.
    \item When an object implements different behaviours, one should
      use interfaces.
  \end{itemize}
\end{frame}
\end{document}
