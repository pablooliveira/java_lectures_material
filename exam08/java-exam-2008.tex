\documentclass[correction]{exercices}
\usepackage{ae, aeguill, graphicx}
\usepackage{fullpage}
\usepackage{xspace}
\usepackage{color}
\usepackage{amsmath, amssymb}
\usepackage{latexsym}
\usepackage{url}
\usepackage{tikz}
\usetikzlibrary{trees}
\usepackage{verbatim}
\renewcommand{\labelenumi}{(\alph{enumi})}

\usepackage{multicol}
\usepackage{listings}
\usepackage{verbatim}

\lstset{language=java, showstringspaces=false}


\begin{document}

\sujet{Introduction to Java MsC - Exam - 2008}

\begin{center}
Documents and notes allowed. All electronic devices are forbidden.
\\Answer on a separate sheet and don't forget to put your name on it!
\\45 minutes advised for this part of the exam.
\end{center}

\section{Multiple choice questions (9 points)}

  There is only \textbf{one} correct answer per question, the grading is :
  
  \begin{tabular}{rl}
  $1$    &point for a correct answer.\\ 
  $-0.5$ &points for a wrong answer.\\
  $0$    &points for a unanswered question.\\
  \end{tabular}
\vspace{1cm}

\begin{question}
A \textbf{final} method
\begin{multicols}{2}
\begin{enumerate}
\item cannot be invoked 
\correct{cannot be overriden}
\item must be overriden
\item must be static
\end{enumerate}
\end{multicols}
\end{question}

\begin{question}
Which one of these propositions is \textbf{true} ?
\begin{enumerate}
\item The method main can be declared protected. 
\correct{A private method can only be called from inside its own class.} 
\item A method with no modifier can be called from a subclass.
\item Private methods cannot be static.
\end{enumerate}
\end{question}

\begin{question}
Which one of these propositions is \textbf{false} ?
\begin{enumerate}
\correct{A subclass of an abstract class must be concrete.}
\item An abstract class can implement an interface.
\item An abstract class can have static methods.
\item An abstract class can have final methods.
\end{enumerate}
\end{question}

\begin{question}
Which one of these propositions is \textbf{false} ?
\begin{enumerate}
\item An interface can extend another interface.
\correct{An interface can provide bodies for some of its methods.}
\item A class can implement more than one interface.
\item An interface can be used as a type. 
\end{enumerate}
\end{question}

\begin{question}
\hfill

\begin{verbatim}
class A {
  void foo() {System.out.print("A");}
}
class B extends A {
  void foo() {System.out.print("B");}
}
class C {
  public static void main(String args[]) {
    A a = new A();
    B b = new B();
    A c = b;
    a.foo();
    b.foo();
    c.foo();
  }
}
\end{verbatim}

What is printed when we run class C ?
\begin{multicols}{2}
\begin{enumerate}
\item nothing because it produces an error
\correct{ABB}
\item ABA
\item AAA
\end{enumerate}
\end{multicols}
\end{question}


\begin{question}
\hfill

\begin{verbatim}
class A {
  A () {System.out.print("A");}
  A (int i) {System.out.print("A" + i);}
}
class B extends A {
  B (int i) {System.out.print("B" + i);}
}
class C extends B {
  C (int i) {super(i);i++;System.out.print("C"+i);}
  C () {this(1);}
}
\end{verbatim}

What is printed when we call \lstinline!new C();! ?
\begin{multicols}{2}
\begin{enumerate}
\item B1AC2
\item A1B1C2
\item B1C2A
\correct{AB1C2}
\end{enumerate}
\end{multicols}
\end{question}

\begin{question} \hfill
\begin{verbatim}
public class A {
  static void foo(int i, String[] st) {
    i--;
    st[i] = "bye";
  }
  public static void main(String args[]) {
    int i = 1;
    String greetings[] = {"hello","bye"}; 
    foo(i,greetings);
    System.out.println(greetings[0] + greetings[1] + i);
  }
}
\end{verbatim}

What does this program prints ?
\begin{multicols}{2}
\begin{enumerate}
\correct{byebye1}
\item byebye0
\item hellohello1
\item nothing because it produces an error
\end{enumerate}
\end{multicols}
\end{question}

\begin{question} \hfill
\begin{verbatim}
public class A {
  void bar() throws Exception {
    System.out.println("A");
    throw new Exception();
  }
  void foo() {
    try {
      System.out.println("C");
      bar();
      System.out.println("D");
    } catch (Exception e) {
      System.out.println("E");
      throw e;
    }
  }
  public static void main(String args[]) {
    A a = new A();
    a.foo();
  }
}
\end{verbatim}

What does this program prints ?
\begin{multicols}{2}
\begin{enumerate}
\item CAE
\item CADE
\item CAD
\correct{nothing because it does not compile}
\end{enumerate}
\end{multicols}
\end{question}

\begin{question} \hfill
\begin{verbatim}
String a = "hello";
String b = "hell";
String c = b + "o";
String d = a ;
\end{verbatim}

Which one of these propositions is \textbf{true} ?

\begin{multicols}{2}
\begin{enumerate}
\item a == c 
\correct{a.equals(c)};
\item d == c;
\item c == d;
\end{enumerate}
\end{multicols}
\end{question}

\pagebreak
\section{Tree operations (7 points)}

The class below models a binary tree. 
A binary tree is composed of nodes.
Each node has either two children or none at all.
A node without childrens is called a \textbf{leaf}.

Read and understand this class,  here is a drawing of the
Tree T that is constructed in main:
\begin{figure}[h]
\begin{tikzpicture}
  \node at(0,0) {0}
    child{node{5}}
    child{node{4}
      child{node{1}}
      child{node{2}}};
\end{tikzpicture}
\end{figure}

\begin{verbatim}
class Tree {
  int nodeValue;
  boolean leaf;
  Tree left;
  Tree right;

  Tree(int nodeValue) {
    this.nodeValue = nodeValue;
    leaf = true;
  }

  Tree(int nodeValue, Tree left, Tree right) {
    this.nodeValue = nodeValue;
    this.left = left;
    this.right = right;
    leaf = false;
  }

  int getSum() {...}
  boolean contains(int value) {...}

  public static void main(String args[]) {
    Tree T = new Tree(0, new Tree(5), new Tree(4, new Tree(1), new Tree(2)));
    System.out.println("Sum of all the nodeValues = " + T.getSum());
    System.out.println("Is value 5 in the tree ? " + T.contains(5));
  }
}
\end{verbatim}

\begin{question} (\textbf{3.5 points})\\
Implement the body of method \lstinline!int getSum()!.
This method should return the sum of all the node values of the calling tree.
(For T it would be $12$).

\begin{correction}
\begin{verbatim}
int getSum() {
    if (leaf)
      return nodeValue;
    else
      return nodeValue + left.getSum() + right.getSum();
  }
\end{verbatim}
\end{correction}
\end{question}

\begin{question} (\textbf{3.5 points}) \\
Implement the body of method \lstinline!boolean contains(int value)!.
This method returns \lstinline!true! if and only if value is equal to one
of the nodeValues in the tree.

\begin{correction}
\begin{verbatim}
  boolean contains(int value) {
    if (leaf) 
      return nodeValue == value;
    else
      return left.contains(value) || right.contains(value);
  }
\end{verbatim}
\end{correction}
\end{question}

\section{Summing vectors (4 points)}

The class Vector belows models a vector of integer values.

Complete methods length and sumWith.

Method length should return the number of elements in the current vector.

Method sumWith should compute the sum element by element of the current 
vector with another one, keeping the results in the current one.
Method sumWith should throw NotTheSameLength exception if vectors are of
different size.

\begin{verbatim}
class NotTheSameLength extends Exception {}
class Vector {
  int [] values;
  Vector(int [] values) {this.values = values;}
  int length(){...}
  void sumWith(Vector anotherVector) throws NotTheSameLength {...}

  public static void main(String args[]) {
    try {
      Vector V1 = new Vector(new int[]{1,2,3,4,5});
      Vector V2 = new Vector(new int[]{6,4,2,8,1});
      V1.sumWith(V2);
      /* here V1 values should be {7, 6, 5, 12, 6} */
    }
    catch (NotTheSameLength e) {
      System.out.println("Vectors must have the same length!");
    }
  }
}
\end{verbatim}


\begin{correction}
\begin{verbatim}
  int length(){return this.values.length;}

  void sumWith(Vector anotherVector) throws NotTheSameLength 
  {
    if (this.length() != anotherVector.length()) throw new NotTheSameLength();
    for (int i=0;i<this.length();i++) {
      this.values[i] += anotherVector.values[i];
    }
  }
\end{verbatim}
\end{correction}

\end{document}
