\documentclass[correction]{exercices}
\usepackage{ae, aeguill, graphicx}
\usepackage{fullpage}
\usepackage{xspace}
\usepackage{color}
\usepackage{amsmath, amssymb}
\usepackage{latexsym}
\usepackage{url}
\usepackage{tikz}
\usetikzlibrary{trees}
\usepackage{verbatim}
\renewcommand{\labelenumi}{(\alph{enumi})}

\usepackage{multicol}
\usepackage{listings}
\usepackage{verbatim}

\lstset{language=java, showstringspaces=false}


\begin{document}

\sujet{Introduction to Java MsC - Exam - 2010}

\begin{center}
Documents and notes allowed. All electronic devices are forbidden.
\\Answer on a separate sheet and don't forget to put your name on it!
\\45 minutes advised for this part of the exam.
\end{center}

\section{Exercices (8 points)}

\begin{question} \textbf{(2 points)}
\begin{verbatim}
String a = "hel";
String b = "hel";
String c = "lo";
String d = a;
a = a + c;
b += c;
\end{verbatim}

Which predicate(s) are \textbf{true}, after
executing the previous code block ?

\begin{multicols}{2}
\begin{enumerate}
\item a == b
\item b == a
\item d == a
\item d.equals(a)
\item a.equals(c)
\correct{b.equals(a)}
\correct{a.equals(b)}
\end{enumerate}
\end{multicols}
\end{question}


\begin{question} \textbf{(3 points)}
\begin{verbatim}
class A {
  A () {System.out.print("A");}
  A (int i) {System.out.print("A" + i);}
}
class B extends A {
  B (int i) {this();System.out.print("B" + i);}
  B () {System.out.print("B");}
}
class C extends B {
  C (int i) {super(i+1);i++;System.out.print("C"+i);}
}
public class Test {
  public static void main(String args[]) {
    new C(2);
  }
}
\end{verbatim}

When executed, what does class \verb!Test! print ?
\begin{correction}
ABB3C3
\end{correction}
\end{question}

\begin{question} \textbf{(3 points)}
\begin{verbatim}
public class Test {
  static int secret(int a[], int v) {
    int i = 0;
    while(v < a.length) {
      a[v] = a[v] - 1;
      v ++;
    }
    return v;
  }
  public static void main(String args[])
  {
    int a [] = {4,2,1,6,9,2};
    int v = secret(a,3);
    for (int i=0; i < a.length; i++) {
      System.out.println(a[i]);
    }
    System.out.println(v);
  }
}
\end{verbatim}

When executed, what does class \verb!Test! print ?
\begin{correction}
4
2
1
5
8
1
6
\end{correction}
\end{question}

\section{Problem: Lists and Sets (9 points)}
You are provided with class \verb!SimpleList! which
implements a list of \verb!int! elements.
\begin{verbatim}
class SimpleList {
  int length();
  int getAt(int pos);
  void insertBefore(int pos, int newEl);
}
\end{verbatim}

To manipulate a \verb!SimpleList! you can only use three methods:
\begin{itemize}
  \item \verb!int length()!, which returns the number of elements
    in the list.
  \item \verb!int getAt(int pos)!, which returns the
    element inside the list at position \verb!pos!.
    The position of the first element of list \verb!L! is $0$ and the
    position of the last element is $(\texttt{L.length()} - 1)$.
    If \verb!pos! is smaller than $0$ or greater than $\texttt{L.length()}-1$
    then the method returns $0$.
  \item \verb!int insertBefore(int pos, int newEl)!, which inserts an
    element, \verb!newEl!, at position \verb!pos! and shifts the old
    element (if any) to the right.
    If \verb!pos! is smaller than $0$ or greater than $\texttt{L.length()}$
    then the method does nothing.

    Let's look at an example, suppose we start with an empty \verb!SimpleList!,
\begin{verbatim}
  SimpleList L = new SimpleList();
  L.insertBefore(0, 5);  /* now the list contains a single element <5>     */
  L.insertBefore(0, 4);  /* now the list contains two elements <4,5>       */
  L.insertBefore(2, 7);  /* now the list contains three elements <4,5,7>   */
  L.insertBefore(2, -3); /* now the list contains four elements <4,5,-3,7> */
  L.insertBefore(5, 2);  /* this does nothing                              */
\end{verbatim}
\end{itemize}

An integer set is a collection of unique integers: all the elements
in the set are different. One way of implementing a set
is by using a list where the elements are always inserted
in increasing order (that is to say, given two
positions $\texttt{pos1} < \texttt{pos2}$ in the list \verb!L!,
the predicate $\texttt{L.getAt(pos1)} < \texttt{L.getAt(pos2)}$ is
always true).

You are provided with the class skeleton below, that implements
method \verb!size! which returns the number of elements in the set,
\begin{verbatim}
  class OrderedSet {
    private SimpleList list;
    OrderedSet() {
      this.list = new SimpleList();
    }
    int size() {return this.list.length();}
    void print() {
      ...
    }
    void insert(int newEl) {
      ...
    }
    static OrderedSet intersection(OrderedSet o1, OrderedSet o2) {
      ...
    }
  }
\end{verbatim}

\begin{question} \textbf{Printing a set} (\textbf{1.5 points}) \\
  Write the method \verb!void print()! that should print all the elements
  in the set using the method \verb!System.out.println!.
\begin{correction}
\begin{verbatim}
    void print() {
       for (int i = 0; i < this.size(); i++)
           System.out.println(this.list.getAt(i));
    }
\end{verbatim}
\end{correction}
\end{question}

\begin{question} \textbf{Insertion in sorted order} (\textbf{3.5 points}) \\
  Write the method \verb!void insert(int newEl)! that should insert the new
  element, \verb!newEl!, in the private list so that the elements in the private
  list stay in increasing order. If the element we are trying to insert already
  belongs to the list, this method does nothing. (You are not required to
  optimize for efficiency.)
\begin{correction}
\begin{verbatim}
void insert(int newEl) {
  int i = 0;
  for (; i < this.size(); i++) {
    if (list.getAt(i) == newEl) return;
    if (list.getAt(i) > newEl) break;
  }
  list.insertBefore(i, newEl);
}

OR

void faster_insert(int newEl) {
  int low = 0;
  int high = this.size()-1;

  while(low <= high) {
    int mid = (high+low)/2;
    if (list.getAt(mid) == newEl) return;
    else if (list.getAt(mid) < newEl) low = mid + 1;
    else high = mid - 1;
  }
  list.insertBefore(low, newEl);
}
\end{verbatim}
\end{correction}
\end{question}

\begin{question} \textbf{Intersection of two sets} (\textbf{4 points}) \\
  Write the method
  \verb!static OrderedSet intersection(OrderedSet o1, OrderedSet o2)!
  that should compute the intersection of two sets. In other words, this
  method should return a new OrderedSet which contains all the elements
  that belong to \verb!o1! \textbf{and} to \verb!o2!.
\begin{correction}
\begin{verbatim}
static OrderedSet intersection(OrderedSet o1, OrderedSet o2) {
  OrderedSet newO = new OrderedSet();
  int i = 0, j = 0;
  while(i < o1.size() && j < o2.size()) {
    if (o1.list.getAt(i) < o2.list.getAt(j)) i++;
    else if (o1.list.getAt(i) > o2.list.getAt(j)) j++;
    else {
      newO.insert(o1.list.getAt(i));
      i++; j++;
    }
  }
  return newO;
}
\end{verbatim}
\end{correction}
\end{question}

\end{document}
