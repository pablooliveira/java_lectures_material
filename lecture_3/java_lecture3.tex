\documentclass[10pt,handout]{beamer}
\usepackage{xmpincl}
\includexmp{license}
\usepackage[T1]{fontenc}
\usepackage[latin1]{inputenc}
\usepackage{ae}
\usepackage[french]{babel}
\usepackage{array, longtable}
\usepackage{tikz}
\usetikzlibrary{arrows,shapes,trees,fit,decorations.pathmorphing,backgrounds}
\usetheme{Antibes}
\setbeamertemplate{navigation symbols}{}

% for printing
\usepackage{pgfpages}
\pgfpagesuselayout{2 on 1}[a4paper,border shrink=5mm]
%\pgfpagesuselayout{resize to}[a4paper,border shrink=5mm,landscape]


\usepackage{listings}
\usepackage{verbatim}
\makeatletter
\newwrite\lstvrb@out
\newenvironment{code}{%
  \begingroup
  \@bsphack
  \immediate\openout\lstvrb@out\jobname.lst
  \let\do\@makeother\dospecials\catcode`\^^M\active
  \def\verbatim@processline{%
    \immediate\write\lstvrb@out{\the\verbatim@line}}%
  \verbatim@start}{%
  \immediate\closeout\lstvrb@out
  \@esphack
  \endgroup
  
  \begin{alertblock}{}
    \lstinputlisting[language=java]{\jobname.lst}
  \end{alertblock}}
\makeatother 

\lstset{language=java, basicstyle=\small, commentstyle=\color{blue}\textrm}

\title{Computer Science Introductory Course MSc - Introduction to Java}

\subtitle{Lecture 3: Exceptions, Generics, Collections}

\author[Pablo Oliveira]{Pablo Oliveira \texttt{<pablo@sifflez.org>}}
\institute{ENST}

\date{}

\begin{document}

\begin{frame}
  \titlepage
\end{frame}

\section{Exceptions}
\begin{frame}
  \frametitle{Outline}
  \tableofcontents
\end{frame}


\begin{frame}[fragile]
  \frametitle{Exceptions}
  \begin{definition}
    An exception is an event that indicates an abnormal condition disrupting the normal flow of the program.
  \end{definition}
  \begin{itemize}
    \item Exception is a subclass of java.lang.Throwable.
    \item When an exception is thrown the program is interrupted, the exception is propagated rewinding the call stack until an appropriate exception handler is found.
  \end{itemize}
\end{frame}

\begin{frame}[fragile]
  \frametitle{Creating and throwing an Exception}
  \begin{code}
    class MyException extends Exception {
      MyException() {}
      MyException(String message) {}
    }

    throw new MyException();
  \end{code}
\end{frame}

\begin{frame}[fragile]
\frametitle{Catching an Exception}
\begin{code}
  try{
    // code that can throw an exception 
  } 
  catch (MyException1 e) {
    // treatment in case of Exception1
  }
  catch (MyException2 e) {
    // treatment in case of Exception2
  }
  finally {
    // always execute at the end of the try block
  }
\end{code}
\begin{itemize}
  \item finally is optional.
\end{itemize}
\end{frame}

\begin{frame}[fragile]
\frametitle{Exception rules}
\begin{itemize}
  \item If a method is thrown inside a method, one these two must be true:
  \begin{itemize}
    \item the exception is catched inside a try/catch block.
    \item the method is declared to throw the exception (using keyword \verb!throws!)
  \end{itemize}
  \item One can throw an exception inside a catch block.
  \item One can catch a group of exceptions using an Exception superclass.
\end{itemize}
\end{frame}

\begin{frame}[fragile]
\frametitle{Example}
\begin{code}
  class Example {
    void foo() throws MyException {
      try {
        throw new MyException();
      } catch (MyException e) {
        throw e;
      }
    }
    void bar() {
      try {
        foo();
      } catch(Exception e) {
        System.out.println("Problem!");
      }
    }
  }
\end{code}
\end{frame}

\section{Generics}
\begin{frame}
  \frametitle{Outline}
  \tableofcontents
\end{frame}

\begin{frame}[fragile]
\frametitle{Generics}
\begin{definition}
Generics are a way of instantiating classes or calling methods with a variable type paramater.
This allows to remove unnecessary casts and make safer code, because the compiler knows which type he
is supposed to work with and thus can generate the appropriate checks.
\end{definition}
\end{frame}

\begin{frame}[fragile, shrink]
\frametitle{List without generics}
\structure{Q: We want to implement a list of objects ?}
\begin{code}
  class List {
    private static final int maxSize = 100;
    private int last = 0;
    private Object list[];
    List() {
      list = new Object[maxSize];
    }
    void add(Object o) throws Exception{
      if (last >= maxSize) 
        throw new Exception("no place left");
      list[last++] = o;
    }
    Object get(int index) {
      return list[index];
    }
  }

  List l = new List();
  l.add("Hello");
  String s = (String) l.get(0);
  Integer i = (Integer) l.get(0); // error at runtime!
\end{code}
\end{frame}

\begin{frame}[fragile, shrink]
\frametitle{List with generics}
\begin{code}
  class List<T> {
    private static final maxSize = 100;
    private int last = 0;
    private T list[];
    List() {
      list = (T[]) new Object[maxSize];
    }
    void add(T o) throws Exception{
      if (last >= maxSize) 
        throw new Exception("no place left");
      list[last++] = o;
    }
    T get(int index) {
      return list[index];
    }
  }

  List<String> l = new List<String>();
  l.add("Hello");
  String s = l.get(0);
  Integer i = l.get(0); // detected at compile time!
\end{code}
\end{frame}

\begin{frame}[fragile, shrink]
\frametitle{Generic methods and constructors}
\begin{code}
class Example {
  <T> Example(T something) {}
  <T> void foo(T something) {}
}
  
  Example e = new Example("Hello"); //type inference, 
  e.foo(new Integer(3));            //no need to pass <>
\end{code}
\end{frame}

\section{Collections}
\begin{frame}
  \frametitle{Outline}
  \tableofcontents
\end{frame}

\begin{frame}
  \frametitle{Collections}
\begin{definition}
  Collections framework is a group of classes in java libraries which eases the manipulation of 
  group of objects.
\end{definition}
\begin{itemize}
  \item All the classes in the collection framework implement the Collection interface.
  \item The collection framework provides List, Set, Queue, Map, etc...
  \item The collection classes can be provided with a generic type!
\end{itemize}
\end{frame}

\begin{frame}[fragile]
  \frametitle{Example}
  \begin{code}
    List<String> myList = new ArrayList<String>();
    myList.add("Hello");
    myList.add("Bye");
    myList.get(1);
    java.util.Collection.sort(myList);
  \end{code}
\end{frame}

\begin{frame}[fragile]
  \frametitle{The for-each control construct}
  \begin{code}
    List<String> myList = new ArrayList<String>();
    for (String s : myList) {
      System.out.println(s);
    }
  \end{code}
\end{frame}
\begin{frame}[fragile, plain]
\begin{center}
  \tiny
  This work is licensed under a Creative Commons Attribution-Noncommercial-Share Alike 3.0 Unported License.
\vtop{
  \vskip-3ex
  \hbox{
    \includegraphics[height=0.4cm]{cc-logo} 
  }
}
\end{center}
\end{frame}
\end{document}

