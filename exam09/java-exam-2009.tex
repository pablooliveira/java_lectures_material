\documentclass[]{exercices}
\usepackage{ae, aeguill, graphicx}
\usepackage{fullpage}
\usepackage{xspace}
\usepackage{color}
\usepackage{amsmath, amssymb}
\usepackage{latexsym}
\usepackage{url}
\usepackage{tikz}
\usetikzlibrary{trees}
\usepackage{verbatim}
\renewcommand{\labelenumi}{(\alph{enumi})}

\usepackage{multicol}
\usepackage{listings}
\usepackage{verbatim}

\lstset{language=java, showstringspaces=false}


\begin{document}

\sujet{Introduction to Java MsC - Exam - 2009}

\begin{center}
Documents and notes allowed. All electronic devices are forbidden.
\\Answer on a separate sheet and don't forget to put your name on it!
\\45 minutes advised for this part of the exam.
\end{center}

\section{Exercices (10 points)}

\begin{question} \textbf{(2 points)}
\begin{verbatim}
class Exception1 extends Exception {};
class Exception2 extends Exception1 {};

public class Test {
  static void foo() {
    throw new Exception1();
  }
  static void bar() {
    throw new Exception2();
  }
  public static void main(String args[])
  {
    try {
      foo();
      bar();
    } catch (Exception e) {
      System.out.println("Exception !");
    }
  }
}
\end{verbatim}

The above \verb!Test! class will not compile. Circle the line(s) that produce errors and
explain the nature of the error(s) concisely.
\begin{correction}
On lines 5 and 8, methods \verb!foo! and \verb!bar! should declare they
throw an exception. For example line 5, should read:
\verb!static void foo() throws Exception1 {!.
\end{correction}
\end{question}

\begin{question} \textbf{(2 points)}
\begin{verbatim}
class Animal {};
class Turtle extends Animal {};
public class Test {
  public static void main(String args[])
  {
    Animal a = new Turtle();
    Turtle t = new Turtle();
    t = a;
    a = t;
  }
}
\end{verbatim}

The above \verb!Test! class will not compile. Circle the line(s) that produce errors and
explain the nature of the error(s) concisely.
\begin{correction}
On line 8, \verb!t = a;!, needs a cast since \verb!Turtle! is a subclass of
\verb!Animal!. It should read \verb!t = (Turtle) a;!.
\end{correction}
\end{question}

\begin{question} \textbf{(3 points)}
\begin{verbatim}
class A {
  A () {System.out.print("A");}
  A (int i) {System.out.print("A" + i);}
}
class B extends A {
  B (int i) {System.out.print("B" + i);}
}
class C extends B {
  C (int i) {super(i);i++;System.out.print("C"+i);}
  C () {this(1);}
}
public class Test {
  public static void main(String args[]) {
    new C();
    new B(2);
  }
}
\end{verbatim}

When executed, what does class \verb!Test! print ?
\begin{correction}
AB1C2AB2
\end{correction}
\end{question}

\begin{question} \textbf{(3 points)}
\begin{verbatim}
public class Test {
  static void secret(int a[]) {
    int i = 0;
    int j = a.length-1;
    while(i < j) {
      int t = a[i];
      a[i++] = a[j];
      a[j--] = t;
   }
  }
  public static void main(String args[])
  {
    int a [] = {4,2,1,6,9,2};
    secret(a);
    for (int i=0; i < a.length; i++) {
      System.out.println(a[i]);
    }
  }
}
\end{verbatim}

When executed, what does class \verb!Test! print ?
\begin{correction}
2
9
6
1
2
4
\end{correction}
\end{question}

\section{Problem: Finding an element in a list (10 points)}
You are provided with class \verb!SimpleList! which
implements a list of \verb!int! elements.
\begin{verbatim}
public class SimpleList {
  public int getElementAt(int pos);
  public int length();
}
\end{verbatim}

To manipulate a \verb!SimpleList! you can only use its public methods:
\begin{itemize}
  \item \verb!int length()!, which returns the number of elements
    in the list.
  \item \verb!int getElementAt(int pos)!, which returns the
    element inside the list at position \verb!pos!.
    The position of the first element of list \verb!L! is $0$ and the
    position of the last element is $(\texttt{L.length()} - 1)$.
\end{itemize}

\begin{question} \textbf{Simple search} (\textbf{5 points}) \\
We want to find out wheter a given \verb!int x! belongs to
a \verb!SimpleList L!.

Write in class \verb!SimpleList! a new method,
\begin{verbatim}
  public boolean contains(int x) {
    ...
  }
\end{verbatim}
that returns \verb!true! if \verb!x! belongs to the current \verb!SimpleList!
instance, and that returns \verb!false! if it does not.

\begin{correction}
\begin{verbatim}
  boolean contains(int x) {
    for (int i = 0; i < this.length(); i++) {
      if (this.getElementAt(i) == x) return true;
    }
    return false;
  }
\end{verbatim}
\end{correction}
\end{question}

You are now provided with class \verb!SortedList! that
\verb!extends! class \verb!SimpleList!.
\verb!SortedList! inherits the public methods from class \verb!SimpleList!,
so you manipulate them the same way.
The only difference is that the elements in a \verb!SortedList! are sorted in
increasing order. Given a list \verb!L!, and two positions $ 0 \leq \texttt{pos1} \leq
\texttt{pos2} < \texttt{L.length()}$ you can assume that
$\texttt{L.getElementAt(pos1)} \leq \texttt{L.getElementAt(pos2)}$.

\begin{question} \textbf{Dictionary Search} (\textbf{5 points})\\
We want to find out wheter a given \verb!int x! belongs to
a \verb!SortedList L!.
Since we know that \verb!L! is sorted, we can use a smarter algorithm
than in Question 5:
\begin{enumerate}
  \item Consider the interval $start = 0$ and $end = \texttt{L.length()}-1$.
  \item If $end$  is smaller than $start$, $\texttt{x} \notin \texttt{L}$
  \item Compute $middle$, the center position between $start$ and $end$.
  \item Compare the element at position $middle$ with \verb!x!:
    \begin{itemize}
      \item if \verb!x! is greater: since the list is sorted we know
        that all the elements between $start$ and $middle$ are smaller than
        \verb!x!. There is no point in searching $x$ at the positions
        preceding $middle$. Therefore we set $start = middle+1$, and jump to (b).
      \item if \verb!x! is smaller: we set $end = middle-1$, and
        jump to (b).
      \item if the two values are the same then $\texttt{x} \in \texttt{L}$.
    \end{itemize}
\end{enumerate}

Implement the previous algorithm in class \verb!SortedList! by overriding the method,
\begin{verbatim}
  public boolean contains(int x) {
    ...
  }
\end{verbatim}
that returns \verb!true! if \verb!x! belongs to the current \verb!SortedList!
instance, and that returns \verb!false! if it does not. Your implementation
should use the above dictionary search algorithm.

With this algorithm you reduce significantly the number of tested elements:
in the worst-case the number of tests is $\lceil \log_2(\texttt{L.length()}) \rceil$
elements. For example, given a list of $10$ elements
the dictionary search should not call \verb!getElementAt! more than $4$ times.

\begin{correction}
\begin{verbatim}
  boolean contains(int x) {
    int start = 0;
    int end = this.length()-1;
    while (start <= end) {
      int middle = (end+start)/2;
      int el = this.getElementAt(middle);
      if (x > el) start = middle+1;
      else if (x < el) end = middle-1;
      else return true;
    }
    return false;
  }
\end{verbatim}
\end{correction}
\end{question}

\end{document}
