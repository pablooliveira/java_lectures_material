\documentclass[correction]{exercices}
\usepackage{ae, aeguill, graphicx}
\usepackage{fullpage}
\usepackage{xspace}
\usepackage{color}
\usepackage{amsmath, amssymb}
\usepackage{latexsym}
\usepackage{url}
\usepackage{tikz}
\usepackage{verbatim}
\renewcommand{\labelenumi}{(\alph{enumi})}

\usepackage{listings}
\usepackage{verbatim}

\lstset{language=java, basicstyle=\small, commentstyle=\color{blue}\textrm}


\begin{document}

\sujet{Introduction to Java MsC - Exam - 2008}

Documents and notes allowed. All electronic devices are forbidden.

\section{Multiple choice questions }

\begin{question}
A \textbf{final} method 
\begin{enumerate}
\item cannot be invoked 
\correct{cannot be overriden}
\item must be overriden
\item must be static
\end{enumerate}
\end{question}

\begin{question}
Which one of these propositions is \textbf{true} ?
\begin{enumerate}
\item The method main can be declared protected. 
\correct{A private method can only be called from inside its own class.} 
\item A method with no modifier can be called from a subclass.
\item Private methods cannot be static.
\end{enumerate}
\end{question}

\begin{question}
Which one of these propositions is \textbf{false} ?
\begin{enumerate}
\correct{A subclass of an abstract class must be concrete.}
\item An abstract class can implement an interface.
\item An abstract class can have static methods.
\item An abstract class can have final methods.
\end{enumerate}
\end{question}

\begin{question}
\hfill

\begin{lstlisting}
class A {
  void foo() {System.out.print("A");}
}
class B extends A {
  void foo() {System.out.print("B");}
}
class C {
  public static void main(String args[]) {
    A a = new A();
    B b = new B();
    A c = b;
    a.foo();
    b.foo();
    c.foo();
  }
}
\end{lstlisting}

What is printed when we run class C ?
\begin{enumerate}
\item nothing because this program produces an error
\correct{ABB}
\item ABA
\item AAA
\end{enumerate}
\end{question}


\begin{question}
\hfill

\begin{lstlisting}
class A {
  A () {System.out.print("A");}
  A (int i) {System.out.print("A" + i);}
}
class B extends A {
  B (int i) {System.out.print("B" + i);}
}
class C extends B {
  C (int i) {super(i);i++;System.out.print("C"+i);}
  C () {this(1);}
}
\end{lstlisting}

What is printed when we call \lstinline!new C();! ?
\begin{enumerate}
\item B1AC2
\item A1B1C2
\item B1C2A
\correct{AB1C2}
\end{enumerate}
\end{question}

\begin{question} \hfill
\begin{lstlisting}
public class A {
  static void foo(int i, String[] st) {
    i--;
    st[i] = "bye";
  }
  public static void main(String args[]) {
    int i = 1;
    String greetings[] = {"hello","bye"}; 
    foo(i,greetings);
    System.out.println(greetings[0] + greetings[1] + i);
  }
}
\end{lstlisting}

What does this program prints ?
\begin{enumerate}
\correct{byebye1}
\item byebye0
\item hellohello1
\item nothing because this program produces an error
\end{enumerate}
\end{question}



\end{document}
